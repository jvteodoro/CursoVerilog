%%%%%%%%%%%%%%%%%%%%%%%%%%%%%%%%%%%%%%%%%%%%%%%%%%%%%%%%%%%%%%%
%
% Welcome to Overleaf --- just edit your LaTeX on the left,
% and we'll compile it for you on the right. If you open the
% 'Share' menu, you can invite other users to edit at the same
% time. See www.overleaf.com/learn for more info. Enjoy!
%
%%%%%%%%%%%%%%%%%%%%%%%%%%%%%%%%%%%%%%%%%%%%%%%%%%%%%%%%%%%%%%%
\documentclass{beamer}
\usepackage[utf8]{inputenc}
\usepackage{minted} % Para sintaxe destacada de código
\usepackage{tikz}
%Information to be included in the title page:
\title{Escrita de Módulo}
\author{Sophia Soares Mariano}
\institute{Poliware}
\date{2025}

\begin{document}

\frame{\titlepage}

\begin{frame}
\frametitle{Escrita de Módulo}
A escrita de um módulo em verilog consiste em definir um bloco de código que descreve um circuito digital. 
Pode conter entradas, saídas, e lógica interna, e pode ser instanciado dentro de outros módulos.
\\
\textcolor{red}{A escrita correta do módulo é fundamental para o funcionamento do circuito digital. Dessa forma, preste atenção
aos pequenos detalhes, como ponto e vírgula, parênteses, e a ordem das entradas e saídas.}
\end{frame}


\begin{frame}[fragile]{Escrita básica da headliner do módulo}
\small
\begin{minted}[fontsize=\footnotesize, linenos]{verilog}
module nome_do_modulo (
    input tipo_entrada1, //define o valor como entrada
    input tipo_entrada2,
    output tipo_saida1, //define o valor como saída
    output [31:0] tipo_saida2, //define saída de tamanho diferente
    output reg tipo_saida3 //define o valor como saída do tipo reg
);
endmodule
\end{minted}
Escolha o nome do módulo e seus tipos de entradas e saídas.
\end{frame}

\begin{frame}
\frametitle{Escrita da lógica interna}
Para a lógica interna no módulo, é possível definir variáveis do tipo "wire" ou "reg", criando variáveis internas.
Além disso, é possível utilizar blocos "always" e "assign" para definir o comportamento do circuito.
\end{frame}

\begin{frame}[fragile]{Escrita da lógica interna}
\scriptsize
\begin{minted}[linenos]{verilog}
module nome_do_modulo (
    input tipo_entrada1,
    input tipo_entrada2,
    output tipo_saida1, 
    output reg tipo_saida3 
);
 wire [31:0] saida_interna; // define um wire interno
 reg [31:0] saida_reg; // define um reg interno

    always @(*) begin
        if (tipo_entrada1 > tipo_entrada2) begin
            tipo_saida3 <= tipo_entrada1 + tipo_entrada2; // lógica combinacional
        end
    end

    assign tipo_saida1 = tipo_entrada1 & tipo_entrada2; 
    assign saida_interna = tipo_entrada1 + tipo_entrada2; //para um wire

    always @(posedge clk) begin //para um reg
        saida_reg <= saida_interna; 
    end

endmodule
\end{minted}
\end{frame}

\begin{frame}
\frametitle{Instanciação de outros módulos dentro do módulo}

Vamos apenas mostrar como se usa a instância. Para maiores explicações, acesse a aula de sintaxe e nomenclatura.
\end{frame}

\begin{frame}[fragile]{Instanciação de outros módulos dentro do módulo}
\small
\begin{minted}[linenos]{verilog}
module nome_do_modulo (
    input tipo_entrada1,
    input tipo_entrada2,
    output tipo_saida1, 
    output [31:0] tipo_saida2, 
    output reg tipo_saida3 
);
//conecta os valores do módulo externo com o interno
    nome_do_modulo_interno instancia1 ( 
            .entrada1(tipo_entrada1), 
            .entrada2(tipo_entrada2),
            .saida1(tipo_saida1),
            .saida2(tipo_saida2)
        );
endmodule
\end{minted}
\end{frame}

\begin{frame}[plain]
    \centering
    {\Huge \textbf{Obrigado!}}
\end{frame}

\end{document}