%----------------------------------------------------------------------------------------
%    PACKAGES AND THEMES
%----------------------------------------------------------------------------------------

\documentclass[aspectratio=169,xcolor=dvipsnames]{beamer}
\usetheme{SimpleDarkBlue}

\usepackage{hyperref}
\usepackage{graphicx}
\usepackage{booktabs}
\usepackage{minted}
\usepackage[portuguese]{babel}

%----------------------------------------------------------------------------------------
%    TITLE PAGE
%----------------------------------------------------------------------------------------

\title{Verilog}
\subtitle{Estados de Sinais Especiais}

\author{Matheus Davi Leão}

\institute
{
    Poliware \\
    Escola Politécnica da Universidade de São Paulo
}
\date{\today}

%----------------------------------------------------------------------------------------
%    PRESENTATION SLIDES
%----------------------------------------------------------------------------------------

\begin{document}

\begin{frame}
    \titlepage
\end{frame}

\begin{frame}{Seções}
    \tableofcontents
\end{frame}

%------------------------------------------------
\section{Introdução}
%------------------------------------------------

\begin{frame}{Introdução}
    Em Verilog, além dos valores binários tradicionais (0 e 1), existem outros estados de sinais que podem ser utilizados para representar condições especiais. Esses estados são especialmente úteis em simulações e na modelagem de circuitos digitais complexos.

    Os principais estados especiais são:
    \begin{itemize}
        \item \texttt{x} (indeterminado)
        \item \texttt{z} (alta impedância)
        \item \texttt{?} (don't care)
    \end{itemize}
\end{frame}

%------------------------------------------------
\section{x (Indeterminado)}
%------------------------------------------------

\begin{frame}
    \Huge{\centerline{\textbf{x (Indeterminado)}}}
\end{frame}

\begin{frame}[fragile]{x (Indeterminado)}
O estado \texttt{x} é utilizado para representar um valor indeterminado ou desconhecido. Isso pode ocorrer quando um sinal não foi inicializado ou quando há conflito entre sinais.

\begin{block}{Exemplo}
\begin{minted}{verilog}
reg a;
initial begin
    a = 1'bx; // a é inicializado como indeterminado
end
\end{minted}
\end{block}

\textbf{Características:}
\begin{itemize}
    \item Indica valor desconhecido ou não inicializado.
    \item Muito usado em simulações para detectar problemas de inicialização ou conflitos.
    \item Não deve ser usado em hardware real.
\end{itemize}
\end{frame}

%------------------------------------------------
\section{z (Alta Impedância)}
%------------------------------------------------

\begin{frame}
    \Huge{\centerline{\textbf{z (Alta Impedância)}}}
\end{frame}

\begin{frame}[fragile]{z (Alta Impedância)}
O estado \texttt{z} representa alta impedância, indicando que o fio não está dirigindo nem 0 nem 1 (desconectado).

\begin{block}{Exemplo}
\begin{minted}{verilog}
assign saida = (controle) ? 1'bz : 1'b0;
\end{minted}
\end{block}

\textbf{Características:}
\begin{itemize}
    \item Usado para modelar barramentos compartilhados.
    \item Permite que múltiplos drivers compartilhem o mesmo fio.
    \item Em hardware real, representa um pino "desconectado".
\end{itemize}
\end{frame}

%------------------------------------------------
\section{? (Don't Care)}
%------------------------------------------------

\begin{frame}
    \Huge{\centerline{\textbf{? (Don't Care)}}}
\end{frame}

\begin{frame}[fragile]{? (Don't Care)}
O estado \texttt{?} é utilizado para indicar que o valor de um sinal não é relevante em uma determinada condição, especialmente em instruções de seleção.

\begin{block}{Exemplo}
\begin{minted}{verilog}
casez (entrada)
    4'b10??: saida = 1; // bits menos significativos podem ser 0 ou 1
    default: saida = 0;
endcase
\end{minted}
\end{block}

\textbf{Características:}
\begin{itemize}
    \item Usado em expressões de seleção (\texttt{casez}, \texttt{casex}).
    \item Facilita a descrição de condições onde certos bits são irrelevantes.
    \item Não representa um valor físico, apenas facilita a síntese e simulação.
\end{itemize}
\end{frame}

%------------------------------------------------
\end{document}
