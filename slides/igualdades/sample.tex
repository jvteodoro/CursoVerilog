%----------------------------------------------------------------------------------------
%    PACKAGES AND THEMES
%----------------------------------------------------------------------------------------

\documentclass[aspectratio=169,xcolor=dvipsnames]{beamer}
\usetheme{SimpleDarkBlue}

\usepackage{hyperref}
\usepackage{graphicx} % Allows including images
\usepackage{booktabs} % Allows the use of \toprule, \midrule and \bottomrule in tables
\usepackage{minted}
\usepackage[portuguese]{babel}

%----------------------------------------------------------------------------------------
%    TITLE PAGE
%----------------------------------------------------------------------------------------

\title{Verilog}
\subtitle{Igualdades}

\author{Diego Fontes de Avila}

\institute
{
    Poliware \\
    Escola Politécnica da Universidade de São Paulo % Your institution for the title page
}
\date{\today} % Date, can be changed to a custom date

%----------------------------------------------------------------------------------------
%    PRESENTATION SLIDES
%----------------------------------------------------------------------------------------

\begin{document}

\begin{frame}
    % Print the title page as the first slide
    \titlepage
\end{frame}

%------------------------------------------------
\section{Igualdades}
%------------------------------------------------

\begin{frame}{Igualdades}
    Os dois tipos mais comuns de operadores de igualdade são:
    \begin{table}
        \begin{tabular}{l l}
            \toprule
            \textbf{Operador} & \textbf{Significado} \\
            \midrule
            Igualdade (equal)            & ==               \\
            Desigualdade (not equal)     & !=               \\
            \bottomrule
        \end{tabular}
    \end{table}
    Entretanto, existem também operadores de igualdade específicos para comparação de valores do \textbf{Verilog}, que são:
    \begin{table}
        \begin{tabular}{l l}
            \toprule
            \textbf{Operador} & \textbf{Significado} \\
            \midrule
            Igualdade estrita (strict equality)      & ===               \\
            Desigualdade estrita (strict not equal)  & !==               \\
            \bottomrule
        \end{tabular}
    \end{table}
    Esses operadores se distinguem por considerar o estado dos bits de alta impedância (z) e não inicializados (x) nas comparações.
\end{frame}
\begin{frame}[fragile]{Igualdades}
   Alguns exemplos são:
    \begin{block}{Exemplos:}
        \begin{minted}{verilog}
        4'b1001 === 4'b1001; // devolve 1
        4'b101x === 4'b1011; // devolve 0
        4'b101x == 4'b1011; // devolve x
        4'b101z == 4'b1011; // devolve x
        4'b101x === 4'b101x; // devolve 1
        4'b101z === 4'b101z; // devolve 1
        4'b101z !== 4'b1z00; // devolve 1
        4'b101z != 4'b1z00; // devolve 1
        4'd9 == 4'd9; // devolve 1
        4'd14 != 4'd14; // devolve 0
        \end{minted}
    \end{block}
\end{frame}
%----------------------------------------------------------------------------------------
\end{document}