%----------------------------------------------------------------------------------------
%    PACKAGES AND THEMES
%----------------------------------------------------------------------------------------

\documentclass[aspectratio=169,xcolor=dvipsnames]{beamer}
\usetheme{SimpleDarkBlue}

\usepackage{hyperref}
\usepackage{graphicx}
\usepackage{booktabs}
\usepackage{minted}
\usepackage[portuguese]{babel}

%----------------------------------------------------------------------------------------
%    TITLE PAGE
%----------------------------------------------------------------------------------------

\title{Verilog}
\subtitle{Tipos de Dados}

\author{Diego Fontes de Avila}

\institute
{
    Poliware \\
    Escola Politécnica da Universidade de São Paulo
}
\date{\today}

%----------------------------------------------------------------------------------------
%    PRESENTATION SLIDES
%----------------------------------------------------------------------------------------

\begin{document}

\begin{frame}
    \titlepage
\end{frame}

\begin{frame}{Seções}
    \tableofcontents
\end{frame}

%------------------------------------------------
\section{Introdução}
%------------------------------------------------

\begin{frame}{Introdução}
    Verilog oferece diversos tipos de dados para descrever circuitos digitais de maneira eficiente e precisa. Cada tipo de dado possui características específicas que determinam como os sinais são armazenados, manipulados e conectados dentro de um projeto.

    Compreender as diferenças entre esses tipos é fundamental para escrever códigos corretos e otimizados em Verilog.
\end{frame}

%------------------------------------------------
\section{Wire (Net)}
%------------------------------------------------

\begin{frame}
    \Huge{\centerline{\textbf{Wire (Net)}}}
\end{frame}

\begin{frame}[fragile]{Wire (Net)}
O tipo \texttt{wire} é utilizado para representar conexões físicas entre componentes. Ele é usado para sinais que são dirigidos por portas ou módulos.

\begin{block}{Exemplo}
\begin{minted}{verilog}
wire a;
wire [3:0] bus; // Vetor de 4 bits
\end{minted}
\end{block}

\textbf{Características:}
\begin{itemize}
    \item Não pode ser atribuído dentro de blocos \texttt{always} ou \texttt{initial}.
    \item Deve ser definido por uma atribuição contínua (\texttt{assign}) ou por uma porta.
\end{itemize}
\end{frame}

%------------------------------------------------
\section{Reg}
%------------------------------------------------

\begin{frame}
    \Huge{\centerline{\textbf{Reg}}}
\end{frame}

\begin{frame}[fragile]{Reg}
O tipo \texttt{reg} é utilizado para armazenar valores em blocos \texttt{always}. Ele representa um registrador de armazenamento.

\begin{block}{Exemplo}
\begin{minted}{verilog}
reg b;
reg [7:0] dado; // Vetor de 8 bits
always @(posedge clk) begin
    b <= 1'b1;
    dado <= dado + 1;
end
\end{minted}
\end{block}

\textbf{Características:}
\begin{itemize}
    \item Pode ser atribuído dentro de blocos \texttt{always} ou \texttt{initial}.
    \item Não representa necessariamente um registrador físico, depende da síntese.
\end{itemize}
\end{frame}

%------------------------------------------------
\section{Vetores}
%------------------------------------------------

\begin{frame}
    \Huge{\centerline{\textbf{Vetores}}}
\end{frame}

\begin{frame}[fragile]{Vetores}
Vetores são utilizados para representar múltiplos bits em uma única variável, como barramentos ou palavras de dados.

\begin{block}{Exemplo}
\begin{minted}{verilog}
wire [15:0] endereco;
reg [31:0] acumulador;
\end{minted}
\end{block}

\textbf{Características:}
\begin{itemize}
    \item Podem ser usados tanto com \texttt{wire} quanto com \texttt{reg}.
    \item O índice mais significativo é especificado primeiro: \texttt{[MSB:LSB]}.
\end{itemize}
\end{frame}

%------------------------------------------------
\section{Integer}
%------------------------------------------------

\begin{frame}
    \Huge{\centerline{\textbf{Integer}}}
\end{frame}

\begin{frame}[fragile]{Integer}
O tipo \texttt{integer} é utilizado para armazenar valores inteiros com sinal, geralmente para contadores ou variáveis auxiliares.

\begin{block}{Exemplo}
\begin{minted}{verilog}
integer i;
initial begin
    for (i = 0; i < 10; i = i + 1) begin
        // ...
    end
end
\end{minted}
\end{block}

\textbf{Características:}
\begin{itemize}
    \item Tamanho padrão de 32 bits com sinal.
    \item Usado principalmente em simulação e blocos \texttt{initial} ou \texttt{always}.
\end{itemize}
\end{frame}

%------------------------------------------------
\section{Real}
%------------------------------------------------

\begin{frame}
    \Huge{\centerline{\textbf{Real}}}
\end{frame}

\begin{frame}[fragile]{Real}
O tipo \texttt{real} é utilizado para representar números de ponto flutuante, geralmente em simulações.

\begin{block}{Exemplo}
\begin{minted}{verilog}
real pi;
initial begin
    pi = 3.14159;
end
\end{minted}
\end{block}

\textbf{Características:}
\begin{itemize}
    \item Não pode ser sintetizado (uso apenas em simulação).
    \item Útil para cálculos matemáticos em testbenches.
\end{itemize}
\end{frame}

%------------------------------------------------
\end{document}
