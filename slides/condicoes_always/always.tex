%----------------------------------------------------------------------------------------
%    PACKAGES AND THEMES
%----------------------------------------------------------------------------------------

\documentclass[aspectratio=169,xcolor=dvipsnames]{beamer}
\usetheme{SimpleDarkBlue}

\usepackage{hyperref}
\usepackage{graphicx}
\usepackage{booktabs}
\usepackage{minted}
\usepackage[portuguese]{babel}

%----------------------------------------------------------------------------------------
%    TITLE PAGE
%----------------------------------------------------------------------------------------

\title{Verilog}
\subtitle{Bloco always e Condições}

\author{Matheus Davi Leão}

\institute
{
    Poliware \\
    Escola Politécnica da Universidade de São Paulo
}
\date{\today}

%----------------------------------------------------------------------------------------
%    PRESENTATION SLIDES
%----------------------------------------------------------------------------------------

\begin{document}

\begin{frame}
    \titlepage
\end{frame}

\begin{frame}{Seções}
    \tableofcontents
\end{frame}

%------------------------------------------------
\section{Introdução}
%------------------------------------------------

\begin{frame}{Introdução}
    O bloco \texttt{always} em Verilog é fundamental para descrever o comportamento de circuitos digitais. Ele permite definir blocos de código que serão executados sempre que ocorrer uma mudança em sinais de entrada ou uma condição específica for atendida.

    Pode ser usado para lógica combinacional ou sequencial, dependendo da sensibilidade do bloco.
\end{frame}

%------------------------------------------------
\section{Always Combinacional}
%------------------------------------------------

\begin{frame}
    \Huge{\centerline{\textbf{Always Combinacional}}}
\end{frame}

\begin{frame}[fragile]{Always Combinacional}
Para lógica combinacional, utiliza-se o bloco \texttt{always} com \texttt{@(*)}, garantindo que o bloco seja reavaliado sempre que qualquer sinal de entrada mudar.

\begin{block}{Exemplo}
\begin{minted}{verilog}
always @(*) begin
    saida = entrada1 + entrada2;
end
\end{minted}
\end{block}

\textbf{Características:}
\begin{itemize}
    \item Atualiza a saída sempre que qualquer entrada mudar.
    \item Deve-se usar atribuição \texttt{=}, que é bloqueante.
    \item Ideal para descrever lógica combinacional.
\end{itemize}
\end{frame}

%------------------------------------------------
\section{Always Sequencial}
%------------------------------------------------

\begin{frame}
    \Huge{\centerline{\textbf{Always Sequencial}}}
\end{frame}

\begin{frame}[fragile]{Always Sequencial}
Para lógica sequencial, o bloco \texttt{always} é sensível à borda de subida (\texttt{posedge}) ou descida (\texttt{negedge}) de um sinal de clock.

\begin{block}{Exemplo}
\begin{minted}{verilog}
always @(posedge clk) begin
    saida <= entrada;
end
\end{minted}
\end{block}

\textbf{Características:}
\begin{itemize}
    \item Executado em eventos de clock.
    \item Utiliza atribuição não bloqueante (\texttt{<=}).
    \item Ideal para descrever registradores e máquinas de estado.
\end{itemize}
\end{frame}

%------------------------------------------------
\section{Atribuições Não Bloqueantes}
%------------------------------------------------

\begin{frame}
    \Huge{\centerline{\textbf{Atribuições Não Bloqueantes}}}
\end{frame}

\begin{frame}[fragile]{Atribuições Não Bloqueantes}
O operador \texttt{<=} indica atribuição não bloqueante, fundamental em lógica sequencial para evitar problemas de ordem de atualização.

\begin{block}{Exemplo}
\begin{minted}{verilog}
always @(posedge clk) begin
    a <= b;
    b <= a;
end
\end{minted}
\end{block}

\textbf{Características:}
\begin{itemize}
    \item Todas as atribuições são realizadas ao final do ciclo de clock.
    \item A ordem das atribuições não interfere no resultado.
    \item Recomendado para circuitos sequenciais.
\end{itemize}
\end{frame}

%------------------------------------------------
\end{document}
