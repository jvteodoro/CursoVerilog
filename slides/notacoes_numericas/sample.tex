%----------------------------------------------------------------------------------------
%    PACKAGES AND THEMES
%----------------------------------------------------------------------------------------

\documentclass[aspectratio=169,xcolor=dvipsnames]{beamer}
\usetheme{SimpleDarkBlue}

\usepackage{hyperref}
\usepackage{graphicx} % Allows including images
\usepackage{booktabs} % Allows the use of \toprule, \midrule and \bottomrule in tables
\usepackage{minted}
\usepackage[portuguese]{babel}

%----------------------------------------------------------------------------------------
%    TITLE PAGE
%----------------------------------------------------------------------------------------

\title{Verilog}
\subtitle{Notações Numéricas}

\author{Diego Fontes de Avila}

\institute
{
    Poliware \\
    Escola Politécnica da Universidade de São Paulo % Your institution for the title page
}
\date{\today} % Date, can be changed to a custom date

%----------------------------------------------------------------------------------------
%    PRESENTATION SLIDES
%----------------------------------------------------------------------------------------

\begin{document}

\begin{frame}
    % Print the title page as the first slide
    \titlepage
\end{frame}

\begin{frame}{Seções}
    % Throughout your presentation, if you choose to use \section{} and \subsection{} commands, these will automatically be printed on this slide as an overview of your presentation
    \tableofcontents
\end{frame}

%------------------------------------------------
\section{Geral}
%------------------------------------------------

\begin{frame}{Geral}
    Em \textbf{Verilog}, existem diversas notações numéricas, as quais são utilizadas para representar valores numéricos de maneiras distintas, compactas e eficientes.

    De forma geral, valores literais em \textbf{Verilog} seguem um padrão definido por essa sequência:
    \begin{itemize}
        \item Prefixo que define o tamanho do valor em bits (se deixado em branco, infere-se o menor tamanho, dado o contexto);
        \item Presença (ou não) de s ou S, indicando se o valor é assinado (signed) ou não (unsigned, que é o padrão);
        \item Base/Radix;
        \item Valor numérico representado na base especificada.
    \end{itemize}

    Em seguida, veremos as diferentes notações utilizadas.
\end{frame}

%------------------------------------------------
\section{Notação Binária}
%------------------------------------------------
\begin{frame}
    \Huge{\centerline{\textbf{Notação Binária}}}
\end{frame}
%------------------------------------------------

\begin{frame}[fragile]{Notação Binária}
A notação binária tende a ser um dos formatos mais utilizados na representação numérica em Verilog. O símbolo 'b' ou 'B' é usado para indicar que o valor é binário.

Aqui estão alguns exemplos de notação binária:
    \begin{block}{Exemplos:}
        \begin{minted}{verilog}
        'b10010 // 5 bits, base binária, valor 18
        4'B1010 // 4 bits, base binária, valor 10
        8'b11111111 // 8 bits, base binária, valor 255
        1'B0 // 1 bit, base binária, valor 0
        1'b1 // 1 bit, base binária, valor 1
        \end{minted}
    \end{block}
\end{frame}

%------------------------------------------------
\section{Notação Octal}
%------------------------------------------------
\begin{frame}
    \Huge{\centerline{\textbf{Notação Octal}}}
\end{frame}
%------------------------------------------------

\begin{frame}[fragile]{Notação Octal}
A notação octal é a menos comum no contexto das disciplinas de Sistemas Digitais I e II, mas ainda é importante conhecê-la. O símbolo 'o' ou 'O' é usado para indicar que o valor é octal. Aqui estão alguns exemplos de notação octal:
    \begin{block}{Exemplos:}
        \begin{minted}{verilog}
        'o10 // 4 bits, base octal, valor 8
        4'O12 // 4 bits, base octal, valor 10
        8'o377 // 8 bits, base octal, valor 255
        1'o0 // 1 bit, base octal, valor 0
        1'O1 // 1 bit, base octal, valor 1
        \end{minted}
    \end{block}
Lembrando que, na notação octal, cada dígito pode assumir valores de 0 a 7. Ou seja, o segundo exemplo apresendado representa o valor 10 na base decimal, pois: $1\cdot8^1 + 2\cdot8^0 = 8 + 2 = 10$.
\end{frame}

%------------------------------------------------
\section{Notação Decimal}
%------------------------------------------------
\begin{frame}
    \Huge{\centerline{\textbf{Notação Decimal}}}
\end{frame}
%------------------------------------------------

\begin{frame}[fragile]{Notação Decimal}
A notação decimal é a mais comum e intuitiva, representando valores na base 10. Não é necessário um prefixo específico, pois é o padrão. Aqui estão alguns exemplos de notação decimal:
    \begin{block}{Exemplos:}
        \begin{minted}{verilog}
        10 // 4 bits, base decimal, valor 10
        4'10 // 4 bits, base decimal, valor 10
        8'255 // 8 bits, base decimal, valor 255
        1'0 // 1 bit, base decimal, valor 0
        1'1 // 1 bit, base decimal, valor 1
        \end{minted}
    \end{block}
\end{frame}

%------------------------------------------------
\section{Notação Hexadecimal}
%------------------------------------------------
\begin{frame}
    \Huge{\centerline{\textbf{Notação Hexadecimal}}}
\end{frame}
%------------------------------------------------

\begin{frame}[fragile]{Notação Hexadecimal}
A notação hexadecimal é muito utilizada. Quase tanto quanto as notações binária e decimal, na verdade. O símbolo 'h' ou 'H' é usado para indicar que o valor é hexadecimal. Aqui estão alguns exemplos de notação hexadecimal:
    \begin{block}{Exemplos:}
        \begin{minted}{verilog}
        'hA // 4 bits, base hexadecimal, valor 10
        4'HF // 4 bits, base hexadecimal, valor 15
        8'hFF // 8 bits, base hexadecimal, valor 255
        1'h0 // 1 bit, base hexadecimal, valor 0
        1'H1 // 1 bit, base hexadecimal, valor 1
        \end{minted}
    \end{block}
Lembrando que, na notação hexadecimal, cada dígito pode ir de 0 a F, em que A, B, C, D, E e F representam os valores 10, 11, 12, 13, 14 e 15, respectivamente. Por exemplo, o terceiro exemplo apresentado representa o valor $255$ na base decimal, pois: $F\cdot16^1 + F\cdot16^0 = 15\cdot16 + 15 = 240 + 15 = 255.$
\end{frame}
%------------------------------------------------
\end{document}