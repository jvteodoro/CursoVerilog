%----------------------------------------------------------------------------------------
%    PACKAGES AND THEMES
%----------------------------------------------------------------------------------------

\documentclass[aspectratio=169,xcolor=dvipsnames]{beamer}
\usetheme{SimpleDarkBlue}

\usepackage{hyperref}
\usepackage{graphicx} % Allows including images
\usepackage{booktabs} % Allows the use of \toprule, \midrule and \bottomrule in tables
\usepackage{minted}
\usepackage[portuguese]{babel}

%----------------------------------------------------------------------------------------
%    TITLE PAGE
%----------------------------------------------------------------------------------------

\title{Verilog}
\subtitle{Genvar}

\author{Diego Fontes de Avila}

\institute
{
    Poliware \\
    Escola Politécnica da Universidade de São Paulo % Your institution for the title page
}
\date{\today} % Date, can be changed to a custom date

%----------------------------------------------------------------------------------------
%    PRESENTATION SLIDES
%----------------------------------------------------------------------------------------

\begin{document}

\begin{frame}
    % Print the title page as the first slide
    \titlepage
\end{frame}

%------------------------------------------------
\section{Genvar}
%------------------------------------------------

\begin{frame}[fragile]{Genvar}
Em \textbf{Verilog}, o termo genvar é usado para declarar variáveis de geração, que são utilizadas em estruturas como generate e for generate. Ela funciona de maneira parecida com uma variável de loop, mas é especificamente projetada para ser usada em contextos de geração de hardware.

O genvar não pode armazenar valores durante a simulação, sendo usado somente para controle da geração de instâncias de módulos ou blocos de código.
\end{frame}

\begin{frame}[fragile]{Genvar}
    \begin{block}{Exemplo:}
        \begin{minted}{verilog}
module exemplo_genvar(
    output [3:0] saida
    );
    generate
        genvar i; /* 1 */
        for (i = 0; i < 4; i = i + 1) begin : nome_bloco
            modulo_instancia instancia (
                .entrada(i),
                .saida(saida[i])
            );
        end
    endgenerate
endmodule
        \end{minted}
    \end{block}
\text{}\\[-20px]
\tiny {1. Versões antigas do Verilog necessitavam da declaração do genvar fora de loops (for), mas versões mais recentes permitem que ele seja declarado dentro do loop de geração.}
\end{frame}
%----------------------------------------------------------------------------------------
\end{document}