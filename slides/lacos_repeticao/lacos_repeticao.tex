\documentclass{article}
\usepackage[utf8]{inputenc}
\usepackage[brazil]{babel}
\usepackage{listings}
\usepackage{xcolor}

\lstdefinelanguage{Verilog}{
  morekeywords={module, input, output, reg, wire, integer, always, begin, end, if, else, for, while, repeat, forever},
  sensitive=true,
  morecomment=[l]{//},
  morestring=[b]",
}

\lstset{
  language=Verilog,
  basicstyle=\ttfamily\small,
  keywordstyle=\color{blue},
  commentstyle=\color{gray},
  stringstyle=\color{red},
  showstringspaces=false,
  breaklines=true,
  frame=single,
  tabsize=2
}

\title{Laços de Repetição em Verilog}
\date{}

\begin{document}

\maketitle

\section*{Laços de Repetição}

Assim como em outras linguagens, \textbf{Verilog} também possui estruturas de repetição, que são úteis para executar blocos de código várias vezes. As principais estruturas de repetição em Verilog são:

\begin{itemize}
  \item \texttt{for}
  \item \texttt{while}
  \item \texttt{repeat}
  \item \texttt{forever}
\end{itemize}

\section*{For}

A estrutura \texttt{for} é utilizada para repetir um bloco de código enquanto uma condição for satisfeita. A sintaxe básica é:

\begin{lstlisting}
for (inicializacao; condicao; incremento) begin
    // codigo a ser executado
end
\end{lstlisting}

A inicialização é executada uma vez antes do loop, a condição é verificada antes de cada iteração e o incremento é executado após cada iteração. Aqui está um exemplo:

\begin{lstlisting}
module exemplo(
    input [3:0] entrada,
    output reg [3:0] saida
);
    integer i;
    always @(entrada) begin
        saida = 4'b0000; // valor inicial da saida
        for (i = 0; i < 4; i = i + 1) begin
            if (entrada[i] == 1'b1) begin
                saida[i] = 1'b1; // se o bit i da entrada for 1, seta o bit i da saida para 1
            end
        end
    end
endmodule 
\end{lstlisting}

Neste exemplo, o loop \texttt{for} percorre cada bit da entrada e, se o bit for 1, seta o bit correspondente na saída para 1.

\section*{While}

A estrutura \texttt{while} executa um bloco de código enquanto uma condição for verdadeira. A sintaxe básica é:

\begin{lstlisting}
while (condicao) begin
    // codigo a ser executado
end
\end{lstlisting}

Aqui está um exemplo:

\begin{lstlisting}
module exemplo(
    input [3:0] entrada,
    output reg [3:0] saida
);
    integer i = 0;
    always @(entrada) begin
        saida = 4'b0000; // valor inicial da saida
        while (i < 4) begin
            if (entrada[i] == 1'b1) begin
                saida[i] = 1'b1; // se o bit i da entrada for 1, seta o bit i da saida para 1
            end
            i = i + 1; // incrementa i
        end
        i = 0; // reseta i para uso futuro
    end
endmodule
\end{lstlisting}

Neste caso, o código tem a mesma função do exemplo anterior, mas utiliza a estrutura \texttt{while} para iterar sobre os bits da entrada. O circuito sintetizado será o mesmo, mas o uso de \texttt{for} é preferível em hardware descritivo, pois \texttt{while} pode gerar loops infinitos, além de ser mais verboso.

\section*{Repeat}

A estrutura \texttt{repeat} é utilizada para repetir um bloco de código um número específico de vezes. A sintaxe básica é:

\begin{lstlisting}
repeat (número_de_repeticoes) begin
    // codigo a ser executado
end
\end{lstlisting}

Exemplo:

\begin{lstlisting}
repeat(10) begin
    clk = ~clk; // inverte o sinal do clock 10 vezes
end
\end{lstlisting}

O exemplo acima é um uso clássico de \texttt{repeat}, que é encorajado para testbenches, mas não é recomendado para circuitos sintetizáveis, pois o número de repetições deve ser conhecido em tempo de compilação. Se o valor passado para o \texttt{repeat} for uma variável, o sintetizador não conseguirá gerar o hardware corretamente.

\section*{Forever}

A estrutura \texttt{forever} é utilizada para criar um loop infinito, que continuará executando até que seja interrompido por uma condição externa. A sintaxe básica é:

\begin{lstlisting}
forever begin
    // codigo a ser executado
end
\end{lstlisting}

\texttt{Forever} é utilizado \textbf{APENAS} em testbenches, pois em circuitos sintetizáveis, um loop infinito não faz sentido. Exemplo:

\begin{lstlisting}
forever begin
    clk = ~clk; // inverte o sinal do clock indefinidamente
    #5;         // espera 5 unidades de tempo
end
\end{lstlisting}

\end{document}
