%----------------------------------------------------------------------------------------
%    PACKAGES AND THEMES
%----------------------------------------------------------------------------------------

\documentclass[aspectratio=169,xcolor=dvipsnames]{beamer}
\usetheme{SimpleDarkBlue}

\usepackage{hyperref}
\usepackage{graphicx}
\usepackage{booktabs}
\usepackage{minted}
\usepackage[portuguese]{babel}

%----------------------------------------------------------------------------------------
%    TITLE PAGE
%----------------------------------------------------------------------------------------

\title{Laços de Repetição em Verilog}
\author{Matheus Davi Leão}
\institute{
    Poliware \\
    Escola Politécnica da Universidade de São Paulo
}
\date{\today}

%----------------------------------------------------------------------------------------
%    PRESENTATION SLIDES
%----------------------------------------------------------------------------------------

\begin{document}

\begin{frame}
    \titlepage
\end{frame}

\begin{frame}{Seções}
    \tableofcontents
\end{frame}

%------------------------------------------------
\section{Laços de Repetição}
%------------------------------------------------

\begin{frame}{Laços de Repetição}
Assim como em outras linguagens, \textbf{Verilog} também possui estruturas de repetição, que são úteis para executar blocos de código várias vezes. As principais estruturas de repetição em Verilog são:

\begin{itemize}
  \item \texttt{for}
  \item \texttt{while}
  \item \texttt{repeat}
  \item \texttt{forever}
\end{itemize}
\end{frame}

%------------------------------------------------
\section{For}
%------------------------------------------------

\begin{frame}
    \Huge{\centerline{\textbf{For}}}
\end{frame}

\begin{frame}[fragile]{For}
A estrutura \texttt{for} é utilizada para repetir um bloco de código enquanto uma condição for satisfeita. A sintaxe básica é:

\begin{block}{Sintaxe}
\begin{minted}{verilog}
for (inicializacao; condicao; incremento) begin
    // codigo a ser executado
end
\end{minted}
\end{block}

A inicialização é executada uma vez antes do loop, a condição é verificada antes de cada iteração e o incremento é executado após cada iteração. Aqui está um exemplo:
\end{frame}

\begin{frame}[fragile]{Exemplo de For}
\begin{block}{Exemplo}
\begin{minted}{verilog}
module exemplo(
    input [3:0] entrada,
    output reg [3:0] saida
);
    integer i;
    always @(entrada) begin
        saida = 4'b0000; // valor inicial da saida
        for (i = 0; i < 4; i = i + 1) begin
            if (entrada[i] == 1'b1) begin
                saida[i] = 1'b1; // se o bit i da entrada for 1, seta o bit i da saida para 1
            end
        end
    end
endmodule 
\end{minted}
\end{block}

Neste exemplo, o loop \texttt{for} percorre cada bit da entrada e, se o bit for 1, seta o bit correspondente na saída para 1.
\end{frame}

%------------------------------------------------
\section{While}
%------------------------------------------------

\begin{frame}
    \Huge{\centerline{\textbf{While}}}
\end{frame}

\begin{frame}[fragile]{While}
A estrutura \texttt{while} executa um bloco de código enquanto uma condição for verdadeira. A sintaxe básica é:

\begin{block}{Sintaxe}
\begin{minted}{verilog}
while (condicao) begin
    // codigo a ser executado
end
\end{minted}
\end{block}

Aqui está um exemplo:
\end{frame}

\begin{frame}[fragile]{Exemplo de While}
\begin{block}{Exemplo}
\begin{minted}{verilog}
module exemplo(
    input [3:0] entrada,
    output reg [3:0] saida
);
    integer i = 0;
    always @(entrada) begin
        saida = 4'b0000; // valor inicial da saida
        while (i < 4) begin
            if (entrada[i] == 1'b1) begin
                saida[i] = 1'b1;
            end
            i = i + 1; // incrementa i
        end
        i = 0; // reseta i para uso futuro
    end
endmodule
\end{minted}
\end{block}

Neste caso, o código tem a mesma função do exemplo anterior, mas utiliza a estrutura \texttt{while} para iterar sobre os bits da entrada. O circuito sintetizado será o mesmo, mas o uso de \texttt{for} é preferível em hardware descritivo, pois \texttt{while} pode gerar loops infinitos, além de ser mais verboso.
\end{frame}

%------------------------------------------------
\section{Repeat}
%------------------------------------------------

\begin{frame}
    \Huge{\centerline{\textbf{Repeat}}}
\end{frame}

\begin{frame}[fragile]{Repeat}
A estrutura \texttt{repeat} é utilizada para repetir um bloco de código um número específico de vezes. A sintaxe básica é:

\begin{block}{Sintaxe}
\begin{minted}{verilog}
repeat (número_de_repeticoes) begin
    // codigo a ser executado
end
\end{minted}
\end{block}

Exemplo:
\end{frame}

\begin{frame}[fragile]{Exemplo de Repeat}
\begin{block}{Exemplo}
\begin{minted}{verilog}
repeat(10) begin
    clk = ~clk; // inverte o sinal do clock 10 vezes
end
\end{minted}
\end{block}

O exemplo acima é um uso clássico de \texttt{repeat}, que é encorajado para testbenches, mas não é recomendado para circuitos sintetizáveis, pois o número de repetições deve ser conhecido em tempo de compilação. Se o valor passado para o \texttt{repeat} for uma variável, o sintetizador não conseguirá gerar o hardware corretamente.
\end{frame}

%------------------------------------------------
\section{Forever}
%------------------------------------------------

\begin{frame}
    \Huge{\centerline{\textbf{Forever}}}
\end{frame}

\begin{frame}[fragile]{Forever}
A estrutura \texttt{forever} é utilizada para criar um loop infinito, que continuará executando até que seja interrompido por uma condição externa. A sintaxe básica é:

\begin{block}{Sintaxe}
\begin{minted}{verilog}
forever begin
    // codigo a ser executado
end
\end{minted}
\end{block}

\texttt{Forever} é utilizado \textbf{APENAS} em testbenches, pois em circuitos sintetizáveis, um loop infinito não faz sentido. Exemplo:
\end{frame}

\begin{frame}[fragile]{Exemplo de Forever}
\begin{block}{Exemplo}
\begin{minted}{verilog}
forever begin
    clk = ~clk; // inverte o sinal do clock indefinidamente
    #5;         // espera 5 unidades de tempo
end
\end{minted}
\end{block}
\end{frame}

%------------------------------------------------
\section{Initial}
%------------------------------------------------

\begin{frame}
    \Huge{\centerline{\textbf{Initial}}}
\end{frame}

\begin{frame}[fragile]{Initial}
A estrutura \texttt{initial} é utilizada para executar um bloco de código uma única vez no início da simulação. A sintaxe básica é:
\begin{block}{Sintaxe}
\begin{minted}{verilog}
initial begin
    // codigo a ser executado
end
\end{minted}
\end{block}

\texttt{Initial} é frequentemente utilizado em testbenches para definir condições iniciais ou gerar sinais de teste. Exemplo:
\end{frame}

\begin{frame}[fragile]{Exemplo de Initial}
\begin{block}{Exemplo}
\begin{minted}{verilog}
initial begin
    clk = 0; // inicializa o clock
    reset = 1; // ativa o reset
    #10 reset = 0; // desativa o reset após 10 unidades de tempo
end
\end{minted}
\end{block}
\end{frame}
\end{document}
%------------------------------------------------