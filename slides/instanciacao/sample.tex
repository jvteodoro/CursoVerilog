%----------------------------------------------------------------------------------------
%    PACKAGES AND THEMES
%----------------------------------------------------------------------------------------

\documentclass[aspectratio=169,xcolor=dvipsnames]{beamer}
\usetheme{SimpleDarkBlue}

\usepackage{hyperref}
\usepackage{graphicx} % Allows including images
\usepackage{booktabs} % Allows the use of \toprule, \midrule and \bottomrule in tables
\usepackage{minted}
\usepackage[portuguese]{babel}

%----------------------------------------------------------------------------------------
%    TITLE PAGE
%----------------------------------------------------------------------------------------

\title{Verilog}
\subtitle{Instanciação}

\author{Diego Fontes de Avila}

\institute
{
    Poliware \\
    Escola Politécnica da Universidade de São Paulo % Your institution for the title page
}
\date{\today} % Date, can be changed to a custom date

%----------------------------------------------------------------------------------------
%    PRESENTATION SLIDES
%----------------------------------------------------------------------------------------

\begin{document}

\begin{frame}
    % Print the title page as the first slide
    \titlepage
\end{frame}

%------------------------------------------------
\section{Instanciação}
%------------------------------------------------

\begin{frame}[fragile]{Instanciação}
    A sintaxe básica para instanciar um módulo é a seguinte:

        \begin{block}{}
        \begin{minted}{verilog}
<nome_do_modulo> <nome_da_instancia> #(<lista_de_parametros>)
    (<lista_de_portas>);
        \end{minted}
    \end{block}

    Onde:
    \begin{itemize}
        \item $<nome\_do\_modulo>$ é o nome do módulo que você deseja instanciar;
        \item $<nome\_da\_instancia>$ é o nome que você deseja dar à instância do módulo;
        \item $<lista\_de\_parametros>$ é uma lista opcional de parâmetros que podem ser passados para o módulo, se ele tiver parâmetros definidos;
        \item $<lista\_de\_portas>$ é uma lista de conexões entre as portas do módulo e os sinais do módulo pai.
    \end{itemize}

\end{frame}

\begin{frame}{Instanciação}
    Tanto a $<lista\_de\_parametros>$ quanto a $<lista\_de\_portas>$ podem ser especificadas de duas maneiras:
    \begin{itemize}
        \item Posicional, onde as portas/parâmetros são conectadas por posição: $<nome\_do\_modulo> <nome\_da\_instancia> (sinal1, sinal2, ...)$;
        \item Nomeada, onde as portas/parâmetros são conectadas por nome: $<nome\_do\_modulo> <nome\_da\_instancia> (.porta1(sinal1), .porta2(sinal2), ...)$;
    \end{itemize}
    A conexão posicional é mais comum, mas a conexão nomeada é mais explícita e pode ser mais fácil de entender, especialmente em módulos com muitas portas. Em seguida está um exemplo de instanciação de um módulo chamado $meu\_modulo$, com dois parâmetros par1 e par2, e duas portas, porta1 e porta2, dentro de um módulo pai:
\end{frame}

\begin{frame}[fragile]{Instanciação}
        \begin{block}{Exemplo:}
        \begin{minted}{verilog}
module meu_modulo #(parameter par1 = 8, parameter par2 = 16) (
    input [par1-1:0] porta1,
    output [par2-1:0] porta2
);
    // Implementação do módulo
endmodule

module modulo_pai;
    // Instanciação do módulo meu_modulo
    meu_modulo #(.par1(8), .par2(16)) instancia1 (
        .porta1(sinal_entrada),
        .porta2(sinal_saida)
    );
endmodule
        \end{minted}
    \end{block}
\end{frame}

\begin{frame}{Instanciação}
    Neste exemplo, o módulo $meu\_modulo$ é instanciado dentro do módulo $modulo\_pai$, com os parâmetros $par1$ e $par2$ definidos como $8$ e $16$, respectivamente. As portas $porta1$ e $porta2$ são conectadas aos sinais $sinal\_entrada$ e $sinal\_saida$, respectivamente. Além disso, a declaração dos parâmetros de $meu\_modulo$ já possui valores padrão, que serão utilizados caso não sejam especificados durante a instanciação.
\end{frame}
%----------------------------------------------------------------------------------------
\end{document}
