\documentclass{article}
\usepackage[utf8]{inputenc}
\usepackage[brazil]{babel}
\usepackage{listings}
\usepackage{xcolor}

\lstdefinelanguage{Verilog}{
  morekeywords={module, input, output, reg, always, if, else, begin, end, case, default},
  sensitive=true,
  morecomment=[l]{//},
  morestring=[b]",
}

\lstset{
  language=Verilog,
  basicstyle=\ttfamily\small,
  keywordstyle=\color{blue},
  commentstyle=\color{gray},
  stringstyle=\color{red},
  showstringspaces=false,
  breaklines=true,
  frame=single,
  tabsize=2
}

\title{Condicionais em Verilog}
\date{}

\begin{document}

\maketitle

\section*{Condicionais}

Os principais operadores que utilizamos para condicionais em Verilog são:

\begin{itemize}
  \item \texttt{if} 
  \item \texttt{else}
  \item \texttt{case}
\end{itemize}

\subsection*{if e else}

\texttt{if} e \texttt{else} são utilizados para executar blocos de código com base em comparações lógicas (booleanas). A estrutura básica é:

\begin{lstlisting}
if (condição) begin
    // codigo a ser executado se a condição for verdadeira
end else begin
    // codigo a ser executado se a condição for falsa
end
\end{lstlisting}

Note que, dependendo do caso, não é necessário o uso de um \texttt{else} após um \texttt{if}:

\begin{lstlisting}
module exemplo(
    input entrada,
    output [3:0] saida 
);
    initial begin
        saida = 4'b0000; // valor inicial da saida
    end
    if (entrada == 1'b1) begin
        saida = 4'b0010; // se entrada for 1, saida será 0010
    end
endmodule
\end{lstlisting}

No caso acima, se a entrada for 1, a saída será 0010, caso contrário, a saída permanecerá com o valor inicial de 0000. Portanto, não é necessário um bloco \texttt{else}.

Agora, para o caso abaixo, o uso de \texttt{else} é necessário, pois se não, nosso output \texttt{saida} apresentaria um valor indefinido (metaestado):

\begin{lstlisting}
module exemplo(
    input entrada,
    output [3:0] saida 
);
    reg [3:0] saida_reg;
    always @(entrada) begin
        if (entrada == 1'b1) begin
            saida_reg = 4'b0010; // se entrada for 1, saida sera 0010
        end
        else begin
            saida_reg = 4'b0000; // se entrada for 0, saida sera 0000
        end
    end
endmodule
\end{lstlisting}

\subsection*{case}

O \texttt{case} é utilizado para comparar uma variável com múltiplos valores possíveis, semelhante ao \texttt{switch} em outras linguagens. A estrutura básica é:

\begin{lstlisting}
case (variavel)
    valor1: begin
        // codigo para valor1
    end
    valor2: begin
        // codigo para valor2
    end
    default: begin
        // codigo para caso nenhum valor corresponda
    end
endcase
\end{lstlisting}

Um exemplo simples:

\begin{lstlisting}
module exemplo(
    input [1:0] entrada,
    output reg [3:0] saida
);
    always @(entrada) begin
        case (entrada)
            2'b00: saida = 4'b0001; // se entrada for 00, saida sera 0001
            2'b01: saida = 4'b0010; // se entrada for 01, saida sera 0010
            2'b10: saida = 4'b0100; // se entrada for 10, saida sera 0100
            2'b11: saida = 4'b1000; // se entrada for 11, saida sera 1000
            default: saida = 4'b0000; // para qualquer outro caso, saida sera 0000
        endcase
    end
endmodule
\end{lstlisting}

Nunca se esqueça de declarar um \texttt{default}, mesmo que seu \texttt{case} tenha todas as possibilidades. Isso é uma boa prática para evitar metaestados indesejados. Por exemplo, no código acima, se a sua entrada estiver em um metaestado, a saída será 0000, evitando assim a propagação de um comportamento indefinido.

\end{document}
