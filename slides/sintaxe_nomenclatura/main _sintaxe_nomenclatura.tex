%%%%%%%%%%%%%%%%%%%%%%%%%%%%%%%%%%%%%%%%%%%%%%%%%%%%%%%%%%%%%%%
%
% Welcome to Overleaf --- just edit your LaTeX on the left,
% and we'll compile it for you on the right. If you open the
% 'Share' menu, you can invite other users to edit at the same
% time. See www.overleaf.com/learn for more info. Enjoy!
%
%%%%%%%%%%%%%%%%%%%%%%%%%%%%%%%%%%%%%%%%%%%%%%%%%%%%%%%%%%%%%%%
\documentclass{beamer}
\usepackage[utf8]{inputenc}
\usepackage{minted} % Para sintaxe destacada de código
\usepackage{tikz}
%Information to be included in the title page:
\title{Instanciação: sintaxe e nomenclatura}
\author{Sophia Soares Mariano}
\institute{Poliware}
\date{2025}

\begin{document}

\frame{\titlepage}

\begin{frame}
\frametitle{Instanciação: sintaxe e nomenclatura}
A instanciação é uma forma de acessar um módulo dentro de outro módulo. É muito utilizada para não deixar que o código fique muito grande e repetitivo.
\\
Suponha que já exista um módulo cuja função você deseja utilizar no seu módulo mais recente.
\end{frame}


\begin{frame}[fragile]{Módulo já existente}
\small
\begin{minted}[fontsize=\footnotesize, linenos]{verilog}
module nome_do_modulo_interno ( //  Módulo já existente
    input tipo_entrada1,
    input tipo_entrada2,
    output tipo_saida1
);
    // exemplo de lógica combinacional
    assign tipo_saida1 = tipo_entrada1 & tipo_entrada2; 
endmodule
\end{minted}
Assim, você quer utilizar esse módulo acima dentro de um novo módulo, de forma a reutilizar o código já existente.
\end{frame}

\begin{frame}[fragile]{Módulo novo}
\scriptsize
\begin{minted}[linenos]{verilog}
module nome_do_modulo ( //escreva o módulo normalmente
    input tipo_entrada1,
    input tipo_entrada2,
    output tipo_saida1, 
    output [31:0] tipo_saida2, 
    output reg tipo_saida3 
);
//coloque o nome do módulo que se deseja instanciar e o nome da instância
 nome_do_modulo_interno instancia1 ( 
            //conecte os valores do módulo externo com o interno
            .entrada1(tipo_entrada1), 
            .entrada2(tipo_entrada2), 
            .saida1(tipo_saida1),
        );

endmodule
\end{minted}
Para conectar o módulo antigo ao novo módulo, eles devem estar no mesmo arquivo, ou pasta, dependendo de sua conexão. Obrigatoriamente, deve-se utilizar esse formato para que os valores do módulo JÁ EXISTENTE sejam passados para as variáveis presentes no módulo mais recente.
\end{frame}

\begin{frame}[fragile]{Módulo novo alternativo}
\scriptsize
\begin{minted}[linenos]{verilog}
module nome_do_modulo ( //escreva o módulo normalmente
    input tipo_entrada1,
    output tipo_saida1, 
    output [31:0] tipo_saida2, 
    output reg tipo_saida3 
);
wire variavel_interna2;

 nome_do_modulo_interno instancia1 ( 
            .entrada1(tipo_entrada1), 
            .entrada2(variavel_interna2), //conecte com valor interno
            .saida1(tipo_saida1),
        );

endmodule
\end{minted}
O módulo instanciado também pode passar valores para variáveis criadas internamente no novo módulo, como no exemplo acima.
\end{frame}

\begin{frame}
\frametitle{Complementando a Instanciação}

Além de instanciar, você pode escrever a lógica interna do módulo novo normalmente, adicionando reg's, wire's e operações lógicas livremente.
\end{frame}

\begin{frame}[fragile]{Complementando a Instanciação}
\scriptsize
\begin{minted}[linenos]{verilog}
module nome_do_modulo ( 
    input tipo_entrada1,
    input tipo_entrada2,
    output tipo_saida1, 
    output [31:0] tipo_saida2, 
    output reg tipo_saida3 
);

 nome_do_modulo_interno instancia1 ( 
            .entrada1(tipo_entrada1), 
            .entrada2(tipo_entrada2), 
            .saida1(tipo_saida1),
        );

    // Outra lógica sequencial
always @(*) begin
        if (tipo_entrada1 > tipo_entrada2) begin
            tipo_saida3 <= tipo_entrada1 + tipo_entrada2; 
            // exemplo de lógica combinacional
        end
    end

endmodule
\end{minted}
\end{frame}

\begin{frame}[plain]
    \centering
    {\Huge \textbf{Obrigado!}}
\end{frame}

\end{document}